\label{app:features}
The following list is the result of the implementation of the feature concept.
It serves as a reference for the user of the feature library as well as for the developer.
In order to understand the meaning of a certain feature it is not necessary anymore to recover the extraction procedure from its implementation.

\section{General remarks}

In the feature library the voltage trace is represented by two vectors, the voltage vector \myid{V} and the corresponding time vector \myid{T}.
\emph{Voltage trace indices} are indices of these two vectors.
Some features require additional trace data, this is:

\begin{itemize}
  \item \myid{stim\_start}, the time at the beginning of the stimulus current.
  \item \myid{stim\_end}, the time at the end of the stimulus current.
\end{itemize}

Some features require parameters in order to determine the extraction procedure.

Feature values are represented as vectors as well.
All elementary features which describe properties of action potentials have entries corresponding to one action potential.
Other features have only one entry containing the feature value.
For some features the extraction procedure can \emph{fail}.
In case of failure the vector containing the feature values is \emph{empty} and an error message is issued.
Note that a successful calculation of a feature can also result in an empty vector (e.g. \myid{peak indices} on a trace without action potentials).

\section{Elementary features}

% efeature
% args:
%   feature name
%   namespace
%   identifier
%   type
%   required features
%   required trace data
%   required parameters
%   semantics
%   detailed description / specification
\begin{efeature}
  {peak indices}
  {LibV1}
  {peak\_indices}
  {(index)}
  {none}
  {V}
  {Threshold}
  {The voltage trace indices at the voltage maxima of the peaks}
  {
  v$_0, \ldots, $v$_{n-1} =$ V \\
  FOR $i = 0, \dots, n - 1$ DO \+ \\
    IF v$_i <$ Threshold AND v$_{i+1} >$ Threshold THEN \+ \\
      APPEND i TO upwardsIndices \- \\
    ENDIF \\
    IF v$_i >$ Threshold AND v$_{i+1} <$ Threshold THEN \+ \\
      APPEND i TO downwardsIndices \- \\
    ENDIF \- \\
  ENDFOR \\
  IF length of upwardsIndices $\neq$ length of downwardsIndices THEN \+ \\
    FAIL "Bad trace shape." \- \\
  ENDIF \\
  u$_0, \ldots, $u$_{n-1} =$ upwardsIndices \\
  d$_0, \ldots, $d$_{n-1} =$ downwardsIndices \\
  FOR $j = 0, \ldots, n-1$ DO \+ \\
    APPEND $i$ TO peak\_indices WITH v$_i$ maximal AND $u_j \le i < d_j$ \- \\
  ENDFOR
  }
  Operating on the voltage trace starting at index 0, each upwards crossing of \myid{V} and the value of \myid{Threshold} is considered a peak onset, and each downwards crossing a peak offset respectively. 
  The peak index is the index of the maximum in between.
\end{efeature}
\mybox{remarks:}{
  The usage for experimental traces is not recommended.
  For noisy traces small fluctuations around the value of \myid{Threshold} are counted as peaks.
  Also traces have been observed where the minima between peaks laid above typical values of \myid{Threshold}, sometimes even above the maxima of other peaks.
  This results in flawed peak count.
  These issues have been addressed in \myid{LibV4:peak\_indices}.
  }

\begin{efeature}
  {peak indices (2nd)}
  {LibV4}
  {peak\_indices}
  {(index)}
  {none}
  {V}
  {min spike height\\&threshold}
  {The voltage trace indices at the voltage maxima of the peaks, noise save}
  {
  $\Delta$v$_0, \ldots, \Delta$v$_{n-1} = \Delta$V \\
  APPEND $0$ TO minima \\
  FOR $i = 0, \dots, n - 1$ DO \+ \\
    IF $\Delta$v$_i < 0$ AND $\Delta$v$_{i+1} > 0$ THEN \+ \\
      APPEND $i + 1$ TO minima \- \\
    ENDIF \- \\
  ENDFOR \\
  APPEND $n - 1$ TO minima \\
  min$_0$, \ldots, min$_{n-1} =$ minima \\
  FOR $i = 0, \dots, n - 2$ DO \+ \\
    max$_i = j$ WITH v$_j$ maximal AND min$_i \le j < $ min$_{i+1}$ \\
    h1 = V[max$_i$] - V[min$_i$] \\
    h2 = V[max$_i$] - V[min$_{i+1}$] \\
    IF h1 > min\_spike\_height AND h2 > min\_spike\_height \\
    OR \\
    V[max$_i$] > threshold AND (h1 > min\_spike\_height OR h2 > min\_spike\_height) THEN \+ \\
      APPEND max$_i$ TO peak\_indices \- \\
    ENDIF \- \\
  ENDFOR \\
  }
  The nulls of the first derivative of \myid{V} with a change of sign from $-$ to $+$ are the minima.
  The maxima between adjacent minima are presumable peaks.
  The left-hand (right-hand) height of a peak is the difference of the maximum and the left-hand (right-hand) minimum.
  A peak is kept if boths heights are bigger than \myid{min spike height} \emph{or} the maximum is bigger than \myid{threshold} and one height is bigger than \myid{min spike height}.
\end{efeature}

\begin{efeature}
  {peak voltage}
  {LibV1}
  {peak\_voltage}
  {mV}
  {peak indices}
  {V}
  {none}
  {The voltages at the maxima of the peaks}
  {
  p$_0, \ldots, $p$_{n-1} =$ peak\_indices \\
  FOR $i = 0, \dots, n - 1$ DO \+ \\
    APPEND V[p$_i$] TO peak\_voltage \- \\
  ENDFOR
  }
  Iterating over the \myid{peak indices} peak voltage yields \myid{V} at each index.
  
\end{efeature}

\begin{efeature}
  {peak time}
  {LibV1}
  {peak\_time}
  {ms}
  {peak indices}
  {T}
  {none}
  {The times of the maxima of the peaks}
  {
  p$_0, \ldots, $p$_{n-1} =$ peak\_indices \\
  FOR $i = 0, \dots, n - 1$ DO \+ \\
    APPEND T[p$_i$] TO peak\_time \- \\
  ENDFOR
  }
  Iterating over the \myid{peak indices}, yield \myid{T} at each index.
  
\end{efeature}

\begin{efeature}
  {trace check}
  {LibV1}
  {trace\_check}
  {none}
  {peak time}
  {stim\_start\\&stim\_end}
  { }
  {Causes feature extraction failure with error message when peaks before or after the stimulus are detected}
  {
  pt$_0, \ldots, $pt$_{n-1} =$ peak\_time \\
  FOR $i = 0, \dots, n - 1$ DO \+ \\
    IF pt$_i$ < stim\_start OR pt$_i > 1.05 \cdot$ stim\_end THEN \+ \\
      FAIL "Trace sanity check failed, there were spike outside the stimulus interval." \- \\
    ENDIF\- \\
  ENDFOR \\
  APPEND 0 TO trace\_check
  }
  Iterating over the values \myid{peak time}, yield 0 as long as all the time values are bigger than \myid{stim start} and smaller than \myid{stim end}. 
  Otherwise fail.
  
\end{efeature}

\begin{efeature}
  {ISI values}
  {LibV1}
  {ISI\_values}
  {ms}
  {peak time}
  {none}
  {none}
  {The interspike intervals (i.e. time intervals) between adjacent peaks, starting at the second peak}
  {
  pt$_0, \ldots, $pt$_{n-1} =$ peak\_time \\
  IF $n < 3$ THEN \+ \\
    FAIL "Three spikes required for calculation of ISI\_values." \- \\
  ENDIF
  FOR $i = 2, \dots, n - 1$ DO \+ \\
  APPEND pt$_i$ - pt$_{i-1}$ TO ISI\_values \- \\
  ENDFOR
  }
  
\end{efeature}

\begin{efeature}
  {doublet ISI}
  {LibV1}
  {doublet\_ISI}
  {ms}
  {peak time}
  {none}
  {none}
  {The time interval between the first too peaks}
  { 
  pt$_0, \ldots, $pt$_{n-1} =$ peak\_time \\
  IF $n < 2$ THEN \+ \\
    FAIL "Need at least two spikes for doublet\_ISI." \- \\
  ENDIF \\
  APPEND pt$_1$ - pt$_0$ TO doublet\_ISI \\
  }
  
\end{efeature}

\begin{efeature}
  {burst ISI indices}
  {LibV1}
  {burst\_ISI\_indices}
  {(index)}
  {ISI values\\&peak indices}
  {none}
  {burst factor (optional)}
  {ISI indices of those ISIs which are the beginning of a burst}
  {
  isi$_0, \ldots, $isi$_{n-1} =$ ISI\_values \\
  $c = 0$ \\
  FOR $i = 1, \dots, n - 2$ DO \+ \\
    median = median of $\{$ isi$_c$, \ldots, isi$_{n-1}\}$ \\
    IF isi$_i$ > burst\_factor $\cdot$ median \\
    AND isi$_{i+1}$ < isi$_i$ / burst\_factor THEN \+ \\
      APPEND $i+1$ TO burst\_ISI\_indices \- \\
    ENDIF \\
    $c = i$ \- \\
  ENDFOR
  }
  The median of the ISIs is determined.
  The \myid{burst factor} defaults to 2.
  Each ISI bigger than \myid{burst factor} times median divides two successive bursts, if it is also bigger than the following ISI times \myid{burst factor}.
  
\end{efeature}

\begin{efeature}
  {mean frequency}
  {LibV1}
  {mean\_frequency}
  {Hz}
  {peak time}
  {stim start\\&stim end}
  {none}
  {The mean frequency of the firing rate}
  {
  pt$_0, \ldots, $pt$_{n-1} =$ peak\_time \\
  $c = 0$ \\
  FOR $i = 0, \dots, n - 1$ DO \+ \\
    IF pt$_i \ge$ stim\_start AND pt$_i \le$ stim\_end THEN \+ \\
      $c = c + 1$ \- \\
    ENDIF \- \\
  ENDFOR \\
  APPEND $1000 \cdot c / ($time\_to\_last\_spike $-$ stim\_start$)$ TO mean\_frequency
  }
  Yield the number of peaks divided by the time to the last spike.
  \mybox{remarks:}{
  The resulting value might be unexpected for bursting or irregular spiking cells.
  Regard \myid{ISI CV} and \myid{adaptation index} to assure that the cell is firing uniformly during the stimulus.
  }
\end{efeature}

\begin{efeature}
  {time to first spike}
  {LibV1}
  {time\_to\_first\_spike}
  {ms}
  {peak time}
  {stim start}
  {none}
  {Time from the start of the stimulus to the maximum of the first peak}
  { 
  pt$_0, \ldots, $pt$_{n-1} =$ peak\_time \\
  IF $n < 1$ THEN \+ \\
    FAIL "One spike required for time\_to\_first\_spike." \- \\
  ENDIF \\
  APPEND pt$_0 -$ stim\_start TO time\_to\_first\_spike
  }
  
\end{efeature}

\begin{efeature}
  {min AHP indices}
  {LibV1}
  {min\_AHP\_indices}
  {(index)}
  {peak indices}
  {V\\&T\\&stim end}
  {none}
  {Voltage trace indices at the after-hyperpolarization}
  {
  pi$_0, \ldots, $pi$_{n-1} =$ peak\_indices \\
  IF $n < 1$ THEN \+ \\
    FAIL "At least one spike required for calculation of min\_AHP\_indices." \- \\
  ENDIF \\
  t$_0, \ldots, $t$_{n-1} =$ T \\ 
  end\_index = minimal $i$ WITH t$_i \ge$ stim\_end \\
  IF end\_index > pi$_{n-1} + 5$ THEN \+ \\
    pi' = (pi$_0$, \ldots, pi$_{n-1}$, end\_index) \- \\
  ENDIF \\
  $m$ = length of pi' \\
  FOR $i = 0, \dots, m - 2$ DO \+ \\
    APPEND j TO min\_AHP\_indices  WITH V[j] minimal AND pi'$_i \le j <$ pi'$_{i+1}$ \- \\
  ENDFOR
  }
  Yield the indices at the voltage minima between two peaks.
  For the last peak yield the minimum between the last peak and the end of the stimulus. 
  
\end{efeature}

\begin{efeature}
  {min AHP values}
  {LibV1}
  {min\_AHP\_values}
  {mV}
  {min AHP indices}
  {V\\&T\\&stim end}
  {none}
  {Voltage values at the after-hyperpolarization}
  {
  ahp$_0, \ldots, $ahp$_{n-1} =$ min\_AHP\_indices \\
  FOR $i = 0, \dots, n - 1$ DO \+ \\
    APPEND V[ahp$_i$] TO min\_AHP\_values \- \\
  ENDFOR
  }
  Iterate over \myid{min AHP indices}.
  Yield \myid{V} at every index.
  
\end{efeature}

\begin{efeature}
  {adaptation index}
  {LibV1}
  {adaptation\_index}
  {none}
  {peak time}
  {stim start\\&stim end}
  {spike skipf\\&max spike skip\\&offset (optional)}
  {Normalized average difference of two consecutive ISIs}
  {
  pt$_0, \ldots, $pt$_{n-1} =$ peak\_time \\
  pt$'_0$, \ldots, pt$'_{m-1}$  = $\{$ pt$_i$ | pt$_i \ge$ stim\_start $-$ offset AND pt$_i \le$ stim\_end $+$ offset $\}$ \\
  $k = \min \{$ spike\_skipf $\cdot m$, max\_spike\_skip$\}$ \\
  pt$''_0$, \ldots, pt$''_{l-1}$ = (pt$'_k$, \ldots, pt$'_m$) \\
  IF $l$ < 4 THEN \+ \\
    FAIL "Minimum 4 spike needed for feature [adaptation\_index]." \- \\
  ENDIF \\
  isi$_0$, \ldots, isi$_{j-1} =$ pt$''_1 -$ pt$''_0$, \ldots, pt$''_{l-1} -$ pt$''_{l-2}$ \\
  sub$_0$, \ldots, sub$_{i-1} =$ isi$_1 -$ isi$_0$, \ldots, isi$_{j-1} -$ isi$_{j-2}$ \\
  sum$_0$, \ldots, sum$_{i-1} =$ isi$_1 +$ isi$_0$, \ldots, isi$_{j-1} +$ isi$_{j-2}$ \\
  APPEND $\frac{1}{i-1} \sum_{n=0}^{i-1} \frac{\mathrm{sub}_n}{\mathrm{sum}_n}$ TO adaptation\_index
  }
  All peaks in the time interval of \myid{stim start}$-$\myid{offset} and \myid{stim end}$+$\myid{offset} are regarded, \myid{offset} defaults to zero.
  The adaptation index is zero for a constant firing rate and bigger than zero for a decreasing firing rate:
  \begin{align*}
    A &=  \frac{1}{N - k - 1} \sum_{i = k}^N \frac{\isi_i - \isi_{i-1}}{\isi_i + \isi_{i-1}} \\
    &= \frac{1}{N - k - 1} \sum_{i = k}^N \frac{\tpeak_{i+1} - 2 \tpeak_i + \tpeak_{i-1}}{\tpeak_{i+1} - \tpeak_{i-1}}
  \end{align*}
  with
  \begin{align*}
    \textrm{the interspike intervals:    } & \isi_i = \tpeak_{i+1} - \tpeak_i, \\
    \textrm{the number of peaks:    } & N.
  \end{align*}
  The first $k$ peaks are skipped.
  The parameter \myid{spike skipf} is the fraction of skipped peaks, $k$ is the minimum of \myid{spike skipf} times $N$ and \myid{max spike skip}.
  
\end{efeature}

\begin{efeature}
  {adaptation index 2}
  {LibV1}
  {adaptation\_index2}
  {none}
  {peak time}
  {stim start\\&stim end}
  {offset (optional)}
  {Normalized average difference of two consecutive ISIs}
  {
  pt$_0, \ldots, $pt$_{n-1} =$ peak\_time \\
  pt$'_0$, \ldots, pt$'_{m-1}$  = $\{$ pt$_i$ | pt$_i \ge$ stim\_start AND pt$_i \le$ stim\_end $\}$ \\
  IF $m$ < 4 THEN \+ \\
    FAIL "Minimum 4 spike needed for feature [adaptation\_index]." \- \\
  ENDIF \\
  isi$_0$, \ldots, isi$_{j-1} =$ pt$'_1 -$ pt$'_0$, \ldots, pt$'_{m-1} -$ pt$'_{m-2}$ \\
  sub$_0$, \ldots, sub$_{i-1} =$ isi$_1 -$ isi$_0$, \ldots, isi$_{j-1} -$ isi$_{j-2}$ \\
  sum$_0$, \ldots, sum$_{i-1} =$ isi$_1 +$ isi$_0$, \ldots, isi$_{j-1} +$ isi$_{j-2}$ \\
  APPEND $\frac{1}{i-1} \sum_{n=0}^{i-1} \frac{\mathrm{sub}_n}{\mathrm{sum}_n}$ TO adaptation\_index
  }
  The extraction is identical to the one of \myid{adaptation index} for \myid{spike skipf} equal zero.
  
\end{efeature}

\begin{efeature}
  {spike width 2}
  {LibV1}
  {spike\_width2}
  {ms}
  {min AHP indices}
  {V\\&T}
  {none}
  {The FWHM of each peak}
  {
  pi$_0, \ldots, $pi$_{n-1} =$ peak\_indices \\
  ahp$_0, \ldots, $ahp$_{n-1} =$ min\_AHP\_indices \\
  FOR $i = 0, \dots, n - 1$ DO \+ \\
    onset\_index = $\arg\max_j$ d$^2$V[j] WITH ahp$_i \le j \le$ pi$_{i+1}$ \\
    onset\_voltage = V[onset\_index] \\
    peak\_voltage = V[pi$_{i+1}$] \\
    half\_voltage = (onset\_voltage + peak\_voltage) / 2 \\
    *** rising phase *** \\
    half\_index = $\min j$ WITH V[j] > half\_voltage AND ahp$_i \le j \le$ pi$_{i+1}$ \\
    $t_0 =$ T[half\_index - 1] \\
    $v_0 =$ V[half\_index - 1] \\
    $v_1 =$ V[half\_index] \\
    $\Delta t =$ T[half\_index] - T[half\_index - 1] \\
    $t_1 = t_0 + \frac{\textrm{half\_voltage} - v_0}{v_1 - v_0} \Delta t$ \\
    *** falling phase (buggy) *** \\
    half\_index = $\min j$ WITH V[j] < half\_voltage AND pi$_{i+1} \le j \le $pi$_{i+1}$ \\
    IF half\_index = pi$_{i+1}$ THEN \+ \\
      FAIL "Falling phase of last spike is missing." \- \\
    ENDIF \\
    $t_0 =$ T[half\_index - 1] \\
    $v_0 =$ V[half\_index - 1] \\
    $v_1 =$ V[half\_index] \\
    $\Delta t =$ T[half\_index] - T[half\_index - 1] \\
    $t_2 = t_0 + \frac{\textrm{half\_voltage} - v_0}{v_1 - v_0} \Delta t$ \\
    APPEND $t_2 - t_1$ TO spike\_width2 \- \\
  ENDFOR
  }
  The peak onset is defined as the maximum of the second derivative.
  For the calculation of the full width at half maximum the height of the peak is taken relative to the voltage at peak onset.
  As one peak often contains only a little number of data points, the time vector is linearly interpolated in the rising and the falling flank.
  
\end{efeature}

\begin{efeature}
  {AP width}
  {LibV1}
  {AP\_width}
  {ms}
  {peak indices\\&min AHP indices}
  {V\\&T\\&stim start}
  {Threshold}
  {Width of each peak at the value of \myid{Threshold}}
  {
  ahp$_0, \ldots, $ahp$_{n-1} =$ min\_AHP\_indices \\
  pi$_0, \ldots, $pi$_{m-1} =$ peak\_indices \\
  start\_index = minimal $i$ WITH T[i] $\ge$ stim\_start \\
  ahp$'_0, \ldots, $ahp$'_{m-1} =$ start\_index, ahp$_0, \ldots, $ahp$_{n-1}$ \\
  FOR $j = 0, \dots, m - 2$ DO \+ \\
    onset\_index = minimal $i$ WITH V[i] $\ge$ Threshold AND ahp$'_j \le i <$ ahp$'_{j+1}$ \\
    offset\_index = minimal $i$ WITH V[i] $\le$ Threshold AND pi$_j \le i <$ ahp$'_{j+1}$ \\
    APPEND T[offset\_index] - T[onset\_index] TO AP\_width \- \\
  ENDFOR
  }
  The peak onset (offset) is determined as the upwards (downwards) crossing of the \myid{V} and the value of \myid{Threshold}.
  AP width yields the time difference between peak onset and peak offset.
  
\end{efeature}

\begin{efeature}
  {spike half width}
  {LibV1}
  {spike\_half\_width}
  {ms}
  {min AHP indices\\&peak indices}
  {V\\&T\\&stim start}
  {none}
  {The FWHM of each peak}
  {
  start\_index = minimal $i$ WITH T[i] $\ge$ stim\_start \\
  ahp$_0, \ldots, $ahp$_{n-1} =$ min\_AHP\_indices \\
  pi$_0, \ldots, $pi$_{m-1} =$ peak\_indices \\
  ahp$'_0, \ldots, $ahp$'_{m-1} =$ start\_index, ahp$_0, \ldots, $ahp$_{n-1}$ \\
  FOR $i = 1, \dots, m - 1$ DO \+ \\
    half\_voltage = (pi$_{i-1}$ + ahp$'_i$) / 2 \\ 
    rise\_index = $\min j$ WITH V[j] > half\_voltage AND ahp$'_{i-1} \le j \le$ pi$_{i-1}$ \\
    $\delta v =$ half\_voltage - V[rise\_index] \\
    $\Delta v =$ V[rise\_index] - V[rise\_index - 1] \\
    $\Delta t =$ T[rise\_index] - T[rise\_index - 1] \\
    $\delta t_1 = \Delta t \frac{\delta v}{\Delta v}$ \\
    fall\_index = $\min j$ WITH V[j] < half\_voltage AND pi$_{i-1} \le j \le$ ahp$'_i$  \\
    $\delta v =$ half\_voltage - V[fall\_index] \\
    $\Delta v =$ V[fall\_index] - V[fall\_index - 1] \\
    $\Delta t =$ T[fall\_index] - T[fall\_index - 1] \\
    $\delta t_2 = \Delta t \frac{\delta v}{\Delta v}$ \\
    APPEND T[fall\_index] + $\delta t_1$ - T[rise\_index] + $\delta t_2$ TO spike\_half\_width \- \\
  ENDFOR
  }
  The height of the peak is defined relative to \myid{V} at the subsequent \myid{min AHP index}.
  As one peak often contains only a little number of data points, the time vector is linearly interpolated in the rising and the falling flank.
  
\end{efeature}

\begin{efeature}
  {burst mean frequency}
  {LibV1}
  {burst\_mean\_freq}
  {Hz}
  {burst ISI indices\\&peak time}
  {none}
  {none}
  {The mean frequency during a burst for each burst}
  {
  isi$_0, \ldots, $isi$_{n-1} =$ burst\_ISI\_indices \\
  isi$'_0, \ldots, $isi$'_{m-1} =$ (0, isi$_0, \ldots, $isi$_{n-1}$) \\
  FOR $i = 0, \dots, m - 2$ DO \+ \\
    IF isi$'_{i+1} -$ isi$'_{i}$ = 1 THEN \+ \\
      $\nu$ = 0 \- \\
    ELSE \+ \\
      $\Delta t =$ peak\_time[isi$'_{i+1}$ - 1] - peak\_time[isi$'_i$] \\
      $\nu$ = 1000 (isi$'_{i+1}$ - isi$'_i$ + 1) / $\Delta t$ \- \\
    ENDIF \\
    APPEND $\nu$ TO burst\_mean\_frequency' \- \\
  ENDFOR \\
  $\Delta t =$ peak\_time[last] - peak\_time[isi$'_i$] \\
  $\nu$ = 1000 (length of peak\_time - isi$'_i$) / $\Delta t$ \\
  APPEND $\nu$ TO burst\_mean\_frequency' \\
  burst\_mean\_frequency = $\{\nu|\nu$ IN burst\_mean\_frequency AND $\nu \neq 0\}$
  }
  Iterate over the \myid{burst ISI indices} and yield the number of peaks divided by the length of the burst.
  
\end{efeature}

\begin{efeature}
  {interburst voltage}
  {LibV1}
  {interburst\_voltage}
  {mV}
  {peak indices\\&burst ISI indices}
  {V\\&T}
  {none}
  {The voltage average in between two bursts}
  {
  isi$_0, \ldots, $isi$_{n-1} =$ burst\_ISI\_indices \\
  IF $n$ < 2 THEN \+ \\
    RETURN \- \\
  ENDIF \\
  FOR $i = 0, \dots, n - 1$ DO \+ \\
    start\_index = peak\_indices[isi$_i$ - 1] \\
    t\_start = T[start\_index] + 5 \\
    end\_index = peak\_indices[isi$_i$] \\
    t\_end = T[end\_index] - 5 \\
    start\_index = -1 + minimal $j$ WITH $j \ge$ start\_index AND T[j] > t\_start \\
    end\_index = 1 + maximal $j$ WITH $j$ < end\_index AND T[j] < t\_end \\
    APPEND mean V[$j$] WITH start\_index < $j$ < end\_index TO interburst\_voltage \- \\
  ENDFOR
  }
  Iterating over the \myid{burst ISI indices} determine the last peak before the burst.
  Starting 5 ms after that peak take the voltage average until 5 ms before the first peak of the subsequent burst.
  
\end{efeature}

\begin{efeature}
  {voltage base}
  {LibV1}
  {voltage\_base}
  {mV}
  {none}
  {V\\&T\\&stim start}
  {none}
  {The membrane resting potential}
  {
  start\_time = stim\_start $\cdot$ 0.25 \\
  end\_time = stim\_start $\cdot$ 0.75 \\
  t$_0, \ldots, $t$_{n-1} =$ T \\ 
  FOR $i = 0, \dots, n - 1$ DO \+ \\
     IF t$_i \ge$ start\_time THEN \+ \\
      sum = sum + V[i] \\
      size = size + 1 \- \\
    ENDIF \\
    IF t$_i$ > end\_time THEN \+ \\
      EXIT FOR \- \\
    ENDIF \- \\
  ENDFOR \\
  APPEND sum / size TO voltage\_base \\
  }
  Yield the average voltage during the time interval $\frac{1}{4}$ times \myid{stim start} and $\frac{3}{4}$ times \myid{stim start} well before the stimulus.
  
\end{efeature}

\begin{efeature}
  {AP height}
  {LibV1}
  {AP\_height}
  {mV}
  {peak voltage}
  {none}
  {none}
  {The voltages at the maxima of the peaks}
  {
  p$_0, \ldots, $p$_{n-1} =$ peak\_indices \\
  FOR $i = 0, \dots, n - 1$ DO \+ \\
    APPEND V[p$_i$] TO AP\_height \- \\
  ENDFOR
  }
  Identical to peak voltage.
  \mybox{remarks:}{
  This feature exists for reasons of compatibility.
  I recommend the usage of \myid{peak voltage} instead.
  \myid{AP height} should rather yield the height of the peak relative to the membrane resting potential.
  }
\end{efeature}

\begin{efeature}
  {AP amplitude}
  {LibV1}
  {AP\_Amplitude}
  {mV}
  {AP begin indices\\&peak voltage}
  {V}
  {none}
  {The relative height of the action potential}
  {
  pv$_0, \ldots, $pv$_{n-1} =$ peak\_voltage \\
  b$_0, \ldots, $b$_{n-1} =$ AP\_begin\_indices \\
  FOR $i = 0, \dots, n - 1$ DO \+ \\
    APPEND pv$_i$ - b$_i$ TO AP\_amplitude \- \\
  ENDFOR
  }
  Yield the difference of \myid{peak voltage} and \myid{V} at \myid{AP begin indices} for each peak.
  
\end{efeature}

\begin{efeature}
  {AHP depth abs}
  {LibV1}
  {AHP\_depth\_abs}
  {mV}
  {min AHP values}
  {none}
  {none}
  {Voltage values at the after-hyperpolarization}
  {
  ahp$_0, \ldots, $ahp$_{n-1} =$ min\_AHP\_indices \\
  FOR $i = 0, \dots, n - 1$ DO \+ \\
    APPEND V[ahp$_i$] TO AHP\_depth\_abs \- \\
  ENDFOR
  }
  Identical to \myid{min AHP values}
  \mybox{remarks:}{
  This feature exists for reasons of compatibility.
  I recommend the usage of \myid{min AHP values} instead.
  }
\end{efeature}

\begin{efeature}
  {AHP depth abs slow}
  {LibV1}
  {AHP\_depth\_abs\_slow}
  {mV}
  {peak indices}
  {V\\&T}
  {none}
  {Voltage values at the ``slow'' after-hyperpolarization}
  {
  pi$_0, \ldots, $pi$_{n-1} =$ peak\_indices \\
  IF n < 3 THEN \+ \\
    FAIL "At least 3 spikes needed for AHP\_depth\_abs\_slow and AHP\_slow\_time." \- \\
  ENDIF \\
  (pi$'_0$, \ldots, pi$'_{m-1}$) = (pi$_1$, \ldots, pi$'_{n-2}$) \\
  FOR $i = 0, \dots, m - 1$ DO \+ \\
    start\_time = T[pi$'_i$] + 5 \\
    start\_index = minimal $j$ WITH T[j] $\ge$ start\_time AND p$'_i \le j$ < p$'_{i+1}$ \\
    min\_voltage = $\min_j$ V[j] WITH start\_index $\le j <$ p$'_{i+1}$ \\
    APPEND min\_voltage TO AHP\_depth\_abs\_slow \- \\
  ENDFOR
  }
  Starting at the second peak iterating over \myid{peak indices} find the minimum of \myid{V} between each pair of peaks.
  For each pair of peaks start the search 5 ms after the first peak.
  
\end{efeature}

\begin{efeature}
  {AHP slow time}
  {LibV1}
  {AHP\_slow\_time}
  {none}
  {AHP depth abs slow}
  {V\\&T}
  {none}
  {Relative timing of the ``slow'' after-hyperpolarization}
  {
  pi$_0, \ldots, $pi$_{n-1} =$ peak\_indices \\
  IF n < 3 THEN \+ \\
    FAIL "At least 3 spikes needed for AHP\_depth\_abs\_slow and AHP\_slow\_time." \- \\
  ENDIF \\
  (pi$'_0$, \ldots, pi$'_{m-1}$) = (pi$_1$, \ldots, pi$'_{n-2}$) \\
  FOR $i = 0, \dots, m - 1$ DO \+ \\
    start\_time = T[pi$'_i$] + 5 \\
    start\_index = minimal $j$ WITH T[j] $\ge$ start\_time AND p$'_i \le j$ < p$'_{i+1}$ \\
    min\_index = $\arg\min_j$ V[j] WITH start\_index $\le j <$ p$'_{i+1}$ \\
    APPEND (T[min\_index] - T[pi$'_i$]) / (T[pi$'_{i+1}$] - T[pi$'_i$]) TO AHP\_slow\_time \- \\
  ENDFOR
  }
  Starting at the second peak iterating over \myid{peak indices} find the minimum of \myid{V} between each pair of peaks.
  For each pair of peaks start the search 5 ms after the first peak.
  Yield the time at the minimum divided by the length of the interspike interval.
  
\end{efeature}

\begin{efeature}
  {time constant}
  {LibV1}
  {time\_constant}
  {ms}
  {none}
  {V\\&T\\&stim start\\&stim end}
  {none}
  {The membrane time constant}
  {
  min\_derivative = 0.005 \\
  decay\_length = 10 \\
  min\_time = 70 \\
  start\_index = 10 + minimal $i$ WITH T[i] $\ge$ stim\_start \\
  middle\_index = minimal $i$ WITH T[i] $\ge$ (stim\_start + stim\_end) / 2 \\
  dvdt = $\frac{\Delta \mathrm{V}_i}{\Delta \mathrm{T}_i}$ WITH start\_index $\le i <$ middle\_index \\
  *** find the decay *** \\
  decay\_index = 0 \\
  WHILE THERE IS x IN (dvdt[decay\_index], \ldots, dvdt[decay\_index + 5]) DO \+ \\
    decay\_index = decay\_index + 1 \- \\
  ENDWHILE \\
  IF decay\_index + 5 = length of dvdt - 1 THEN \+ \\
    FAIL "Could not find the decay." \- \\
  ENDIF \\
  *** find the flat *** \\
  $i$ = decay\_index
  WHILE T[$i$] < T[middle\_index] + min\_time DO \+ \\
    IF dvdt[$i$] > - min\_derivative THEN \+ \\
      $j =$ minimal $j$ WITH T[$j$] - T[$i$] > min\_time \\
      mean = mean of (dvdt[$i$], \ldots, dvdt[$j$]) \\
      IF mean > - min\_derivative THEN \+ \\
        EXIT WHILE \- \\
      ENDIF \- \\
    ENDIF \\
    $i = i + 1$ \- \\
  ENDWHILE \\
  flat\_index = $i$ \\
  IF flat\_index - decay\_index < decay\_length THEN \+ \\
    FAIL "Trace fall time too short." \- \\
  ENDIF \\
  (v$_0$, \ldots, v$_{n-1}$) = (v[decay\_index], \ldots, v[flat\_index]) \\
  (t$_0$, \ldots, t$_{n-1}$) = (t[decay\_index], \ldots, t[flat\_index]) \\
  $x = 0.38$ \\
  golden section search: \\
  FOR $i = 0, \dots, n - 1$ DO \+ \\
    logv$_i = \log(v_i - v_{n-1} + x)$ \- \\
  ENDFOR \\
  slope, residuals = fit\_straight\_line(t, logv) \\
  repeat golden section search to minimize residuals \\
  APPEND -1 / slope TO time\_constant \\
  }
  The extraction of the time constant requires a voltage trace of a cell in a hyperpolarized state.
  Starting at \myid{stim start} find the beginning of the exponential decay where the first derivative of \myid{V(t)} is smaller than -0.005 $\frac{\mathrm{V}}{\mathrm{s}}$ in 5 subseequent points.
  The flat subsequent to the exponential decay is defined as the point where the first derivative of the voltage trace is bigger than $-0.005$ and the mean of the follwowing 70 points as well.
  If the voltage trace between the beginning of the decay and the flat includes more than 9 points, fit an exponential decay.
  Yield the time constant of that decay.
  
\end{efeature}

\begin{efeature}
  {voltage deflection}
  {LibV1}
  {voltage\_deflection}
  {mV}
  {none}
  {V\\&T\\&stim start\\&stim end}
  {none}
  {The relative steady state voltage in a hyperpolarized state}
  {
  base = mean of V[i] WITH 0 $\le$ T[i] < stim\_start \\
  end\_index = minimal $i$ WITH t$_i \ge$ stim\_end \\
  ss = mean of V[i] WITH end\_index - 10 < i < end\_index - 5 \\
  APPEND ss - base TO voltage\_deflection
  }
  Calculate the base voltage as the mean of \myid{V} before \myid{stim start}.
  Calculate the steady state voltage as the mean of 5 values of \myid{V} 10 points before \myid{stim end}.
  Yield the difference of steady state voltage and base voltage.
  
\end{efeature}

\begin{efeature}
  {ohmic input resistance}
  {LibV1}
  {ohmic\_input\_resistance}
  {M$\Omega$}
  {voltage deflection}
  {stimulus current}
  {none}
  {The ohmic input resistance $R_\mathrm{in}$ of the cell}
  {
  APPEND voltage\_deflection / stimulus\_current TO ohmic\_input\_resistance
  }
  Yield \myid{voltage deflection} divided by \myid{stimulus current}.
\end{efeature}

\begin{efeature}
  {maximum voltage}
  {LibV1}
  {maximum\_voltage}
  {mV}
  {none}
  {V\\&T\\&stim start\\&stim end}
  {none}
  {The maximum voltage during a stimulus}
  {
  max = maximal V[i] WITH i: stim\_start $\le$ T[i] < stim\_end \\
  APPEND max TO maximum\_voltage
  }
  Find the maximum of \myid{V} between \myid{stim start} and \myid{stim end}.
\end{efeature}

\begin{efeature}
  {steady state voltage}
  {LibV1}
  {steady\_state\_voltage}
  {mV}
  {none}
  {V\\&T\\&stim end}
  {none}
  {Average voltage after the stimulus}
  {
  mean = mean V[i] WITH i: stim\_end < T[i] \\
  APPEND mean TO steady\_state\_voltage
  }
  Yield the average of \myid{V} after \myid{stim end}.
\end{efeature}

\begin{efeature}
  {ISI CV}
  {LibV1}
  {ISI\_CV}
  {none}
  {ISI values}
  {none}
  {none}
  {The coefficient of variation of the ISIs}
  {
  $\mu$ = mean of ISI\_values \\
  $\sigma$ = standard deviation of ISI\_values \\
  APPEND $\frac{\sigma}{\mu}$ TO ISI\_CV
  }
  Yield the coefficient of variation:
  \begin{align*}
    c_v =& \frac{\sigma}{\mu} \\
  \end{align*}
  with
  \begin{align*}
    \textrm{standard deviation:  } \sigma =& \sqrt{\frac{1}{N-1} \sum_{i=1}^N \left(\isi_i - \mu \right)^2}, \\
    \textrm{mean:  } \mu =& \frac{1}{N} \sum_{i=1}^N \isi_i.
  \end{align*}
  
\end{efeature}

\begin{efeature}
  {Spikecount}
  {LibV1}
  {Spikecount}
  {none}
  {peak indices\\&trace check}
  {none}
  {none}
  {The number of peaks during stimulus}
  {
  APPEND length of peak\_indices TO Spikecount
  }
  Yield the length of \myid{peak indices}.
  
\end{efeature}

\begin{efeature}
  {AHP depth}
  {LibV1}
  {AHP\_depth}
  {mV}
  {voltage base\\&min AHP values}
  {none}
  {none}
  {Relative voltage values at the after-hyperpolarization}
  {
  ahp$_0, \ldots, $ahp$_{n-1} =$ min\_AHP\_values \\
  FOR $i = 0, \dots, n - 1$ DO \+ \\
    APPEND ahp$_i$ - voltage\_base TO AHP\_depth \- \\
  ENDFOR
  }
  Iterate over \myid{min AHP values} and yield the difference of the value and \myid{voltage base}.
  
\end{efeature}

\begin{efeature}
  {AP begin indices}
  {LibV2}
  {AP\_begin\_indices}
  {(index)}
  {min AHP indices\\&interpolate}
  {V\\&T\\&stim start\\&stim end}
  {none}
  {Voltage trace indices at the onset of each action potential}
  {
  min\_derivative = 12.0 \\
  start\_index = minimal $i$ WITH T[i] $\ge$ stim\_start \\
  ahp$_0, \ldots, $ahp$_{n-1} =$ min\_AHP\_indices \\
  min$_0,$, \ldots, min$_{m-1}$ = (start\_index, ahp$_0$, \ldots, ahp$_{n-1}$) \\
  IF T[min$_{m-1}$] < stim\_end THEN \+ \\
    end\_index = minimal $i$ WITH t$_i \ge$ stim\_end \\
    APPEND end\_index TO min \- \\
  ENDIF \\
  dvdt = $\Delta$ V \\
  FOR $i$ = 0 TO length of min - 2 DO \+ \\
    IF $x \ge$ min\_derivative FOR ALL $x$ IN (dvdt[min$_i$], \ldots, dvdt[min$_i$ + 5]) THEN \+ \\
      APPEND min$_i$ TO AP\_begin\_indices \- \\
    ENDIF \- \\
  ENDFOR
  }
  Iterate over \myid{min AHP indices}.
  If there is no AHP for the last peak, add the index at \myid{stim end} to the indices.
  Yield the action potential onsets where the first derivative of the voltage trace is higher than $12 \frac{\mathrm{V}}{\mathrm{s}}$, for at least 5 points.
  
\end{efeature}

\begin{efeature}
  {AP rise indices}
  {LibV2}
  {AP\_rise\_indices}
  {(index)}
  {peak indices\\&AP begin indices}
  {V}
  {none}
  {Voltage trace index at the rising flank of each action potential.}
  {
  begin$_0, \ldots, $begin$_{n-1} =$ AP\_begin\_indices \\
  pi$_0, \ldots, $pi$_{n-1} =$ peak\_indices \\
  FOR $i = 0, \dots, n - 1$ DO \+ \\
    half\_voltage = (begin$_i$ + pi$_i$) / 2 \\
    rise\_index = $\arg\min_j$ |V[j] - half\_voltage| WITH begin$_i \le j <$ pi$_i$ \\
    APPEND rise\_index TO AP\_rise\_indices \- \\
  ENDFOR
  }
  Yield the indices of the voltage trace after each \myid{AP begin index} where \myid{V} reaches half the maximum of the amplitude of the action potential.
  The amplitude of the action potential is taken relative to \myid{V} at \myid{AP begin indices}.
  
\end{efeature}

\begin{efeature}
  {AP end indices}
  {LibV2}
  {AP\_end\_indices}
  {(index)}
  {peak indices}
  {V\\&T}
  {none}
  {Voltage trace indices at the offset of each action potential}
  {
  pi$_0, \ldots, $pi$_{n-1} =$ (peak\_indices, length of V - 1) \\
  min\_derivative = -12.0 \\
  dvdt = $\Delta$ V \\
  FOR $i = 0, \dots, n - 2$ DO \+ \\
    end\_index = minimal $j$ WITH pi$_i + 1 \le j <$ pi$_{i+1}$ AND dvdt[$j$] > min\_derivative \- \\
  ENDFOR
  }
  Iterate over \myid{peak indices} and find after each index where the first derivative of the voltage trace exceeds $-12 \frac{\mathrm{V}}{\mathrm{s}}$.
  
\end{efeature}

\begin{efeature}
  {AP fall indices}
  {LibV2}
  {AP\_fall\_indices}
  {(index)}
  {peak indices\\&AP begin indices\\&AP end indices}
  {V}
  {none}
  {Voltage trace index at the falling flank of each action potential}
  {
  begin$_0, \ldots, $begin$_{n-1} =$ AP\_begin\_indices \\
  end$_0, \ldots, $end$_{n-1} =$ AP\_end\_indices \\
  pi$_0, \ldots, $pi$_{n-1} =$ peak\_indices \\
  FOR $i = 0, \dots, n - 1$ DO \+ \\
    half\_voltage = (begin$_i$ + pi$_i$) / 2 \\
    fall\_index = $\arg\min_j$ |V[j] - half\_voltage| WITH pi$_i \le j <$ end$_i$ \\
    APPEND fall\_index TO AP\_fall\_indices \- \\
  ENDFOR
  }
  Yield the indices of the voltage trace after each \myid{peak index} where \myid{V} falls down to half the maximum of the amplitude of the action potential.
  The amplitude of the action potential is taken relative to \myid{V} at \myid{AP begin indices}.
  
\end{efeature}

\begin{efeature}
  {AP duration}
  {LibV2}
  {AP\_duration}
  {ms}
  {AP begin indices\\&AP end indices}
  {T}
  {none}
  {Duration of an action potential from onset to offset}
  {
  begin$_0, \ldots, $begin$_{n-1} =$ AP\_begin\_indices \\
  end$_0, \ldots, $end$_{n-1} =$ AP\_end\_indices \\
  FOR $i = 0, \dots, n - 1$ DO \+ \\
    APPEND T[end$_i$] - T[begin$_i$] TO AP\_duration \- \\
  ENDFOR
  }
  Iterate over \myid{AP begin indices} and return the difference of the time at the \myid{AP end index} and the \myid{AP begin index}.
  
\end{efeature}

\begin{efeature}
  {AP duration half width}
  {LibV2}
  {AP\_duration\_half\_width}
  {ms}
  {AP rise indices\\&AP fall indices}
  {T}
  {none}
  {FWHM of each action potential}
  {
  rise$_0, \ldots, $rise$_{n-1} =$ AP\_rise\_indices \\
  fall$_0, \ldots, $fall$_{n-1} =$ AP\_fall\_indices \\
  FOR $i = 0, \dots, n - 1$ DO \+ \\
    APPEND T[fall$_i$] - T[rise$_i$] TO AP\_duration\_half\_width \- \\
  ENDFOR
  }
  Iterate over \myid{AP rise indices} and return the difference of the time at the \myid{AP fall index} and the \myid{AP rise index}.
  
\end{efeature}

\begin{efeature}
  {AP rise time}
  {LibV2}
  {AP\_rise\_time}
  {ms}
  {AP begin indices\\&peak indices}
  {T}
  {none}
  {Time from action potential onset to the maximum}
  {
  pi$_0, \ldots, $pi$_{n-1} =$ peak\_indices \\
  begin$_0, \ldots, $begin$_{n-1} =$ AP\_begin\_indices \\
  FOR $i = 0, \dots, n - 1$ DO \+ \\
    APPEND T[pi$_i$] - T[begin$_i$] TO AP\_rise\_time \- \\
  ENDFOR
  }
  Iterate over \myid{AP begin indices} and return the difference of the time at the \myid{peak index} and the \myid{AP begin index}.
  
\end{efeature}

\begin{efeature}
  {AP fall time}
  {LibV2}
  {AP\_fall\_time}
  {ms}
  {peak indices\\&AP end indices}
  {T}
  {none}
  {Time from action potential maximum to the offset}
  {
  pi$_0, \ldots, $pi$_{n-1} =$ peak\_indices \\
  end$_0, \ldots, $end$_{n-1} =$ AP\_end\_indices \\
  FOR $i = 0, \dots, n - 1$ DO \+ \\
    APPEND T[end$_i$] - T[pi$_i$] TO AP\_fall\_time \- \\
  ENDFOR
  }
  Iterate over \myid{peak indices} and return the difference of the time at the \myid{end index} and the \myid{peak index}.
  
\end{efeature}

\begin{efeature}
  {AP rise rate}
  {LibV2}
  {AP\_rise\_rate}
  {$\frac{\mathrm{V}}{\mathrm{s}}$}
  {AP begin indices\\&peak indices}
  {V\\&T}
  {none}
  {Voltage change rate during the rising phase of the action potential}
  {
  pi$_0, \ldots, $pi$_{n-1} =$ peak\_indices \\
  begin$_0, \ldots, $begin$_{n-1} =$ AP\_begin\_indices \\
  FOR $i = 0, \dots, n - 1$ DO \+ \\
    APPEND (V[pi$_i$] - V[begin$_i$]) / (T[pi$_i$] - T[begin$_i$]) TO AP\_rise\_rate \- \\
  ENDFOR
  }
  Iterate over \myid{AP begin indices} and return the ratio of the voltage difference and the time difference at the \myid{peak index} and the \myid{AP begin index}.
  
\end{efeature}

\begin{efeature}
  {AP fall rate}
  {LibV2}
  {AP\_fall\_rate}
  {$\frac{\mathrm{V}}{\mathrm{s}}$}
  {peak indices\\&AP end indices}
  {V\\&T}
  {none}
  {Voltage change rate during the falling phase of the action potential.}
  {
  pi$_0, \ldots, $pi$_{n-1} =$ peak\_indices \\
  end$_0, \ldots, $end$_{n-1} =$ AP\_end\_indices \\
  FOR $i = 0, \dots, n - 1$ DO \+ \\
    APPEND (V[end$_i$] - V[pi$_i$]) / (T[end$_i$] - T[pi$_i$]) TO AP\_fall\_rate \- \\
  ENDFOR
  }
  Iterate over \myid{peak indices} and return the ratio of the voltage difference and the time difference at the \myid{AP end index} and the \myid{peak index}.
  
\end{efeature}

\begin{efeature}
  {fast AHP}
  {LibV2}
  {fast\_AHP}
  {mV}
  {AP begin indices\\&min AHP indices}
  {V}
  {none}
  {Voltage value of the action potential onset relative to the subsequent AHP}
  {
  begin$_0, \ldots, $begin$_{n-1} =$ AP\_begin\_indices \\
  ahp$_0, \ldots, $ahp$_{n-1} =$ min\_AHP\_indices \\
  FOR $i = 0, \dots, n - 1$ DO \+ \\
    APPEND V[begin$_i$] - V[ahp$_i$] TO fast\_AHP \- \\
  ENDFOR
  }
  Iterate over \myid{AP begin indices} and yield the difference of \myid{V} at the \myid{AP begin index} and the \myid{min AHP index}.
  
\end{efeature}

\begin{efeature}
  {AP amplitude change}
  {LibV2}
  {AP\_amplitude\_change}
  {none}
  {peak voltage}
  {none}
  {none}
  {Difference of the amplitudes of the second and the first action potential divided by the amplitude of the first action potential}
  {
  amp$_0, \ldots, $amp$_{n-1} =$ AP\_amplitude\_change \\
  FOR $i = 0, \dots, n - 2$ DO \+ \\
    APPEND (amp$_{i+1}$ - amp$_0$) / amp$_0$ TO AP\_amplitude\_change \- \\
  ENDFOR
  }
  
\end{efeature}

\begin{efeature}
  {AP duration change}
  {LibV2}
  {AP\_duration\_change}
  {none}
  {AP duration}
  {none}
  {none}
  {Difference of the durations of the second and the first action potential divided by the duration of the first action potential}
  {
  dur$_0, \ldots, $dur$_{n-1} =$ AP\_duration \\
  FOR $i = 0, \dots, n - 2$ DO \+ \\
    APPEND (dur$_{i+1}$ - dur$_0$) / dur$_0$ TO AP\_duration\_change \- \\
  ENDFOR
  }
  
\end{efeature}

\begin{efeature}
  {AP rise rate change}
  {LibV2}
  {AP\_rise\_rate\_change}
  {none}
  {AP rise rate}
  {none}
  {none}
  {Difference of the rise rates of the second and the first action potential divided by the rise rate of the first action potential}
  {
  rr$_0, \ldots, $rr$_{n-1} =$ AP\_rise\_rate \\
  FOR $i = 0, \dots, n - 2$ DO \+ \\
    APPEND (rr$_{i+1}$ - rr$_0$) / rr$_0$ TO AP\_rise\_rate\_change \- \\
  ENDFOR
  }
  
\end{efeature}

\begin{efeature}
  {AP fall rate change}
  {LibV2}
  {AP\_fall\_rate\_change}
  {none}
  {AP fall rate}
  {none}
  {none}
  {Difference of the fall rates of the second and the first action potential divided by the fall rate of the first action potential}
  {
  fr$_0, \ldots, $fr$_{n-1} =$ AP\_fall\_rate \\
  FOR $i = 0, \dots, n - 2$ DO \+ \\
    APPEND (fr$_{i+1}$ - fr$_0$) / fr$_0$ TO AP\_fall\_rate\_change \- \\
  ENDFOR
  }
  
\end{efeature}

\begin{efeature}
  {fast AHP change}
  {LibV2}
  {fast\_AHP\_change}
  {none}
  {fast AHP}
  {none}
  {none}
  {Difference of the \myid{fast AHP} of the second and the first action potential divided by the \myid{fast AHP} of the first action potential}
  {
  fahp$_0, \ldots, $fahp$_{n-1} =$ fast\_AHP \\
  FOR $i = 0, \dots, n - 2$ DO \+ \\
    APPEND (fahp$_{i+1}$ - fahp$_0$) / fahp$_0$ TO fast\_AHP\_change \- \\
  ENDFOR
  }
  
\end{efeature}

\begin{efeature}
  {AP duration half width change}
  {LibV2}
  {AP\_duration\_half\_width\_change}
  {none}
  {AP duration half width}
  {none}
  {none}
  {Difference of the FWHM of the second and the first action potential divided by the FWHM of the first action potential}
  {
  dhw$_0, \ldots, $dhw$_{n-1} =$ AP\_duration\_half\_width \\
  FOR $i = 0, \dots, n - 2$ DO \+ \\
    APPEND (dhw$_{i+1}$ - dhw$_0$) / dhw$_0$ TO AP\_duration\_half\_width\_change \- \\
  ENDFOR
  }
  
\end{efeature}

\begin{efeature}
  {steady state hyper}
  {LibV2}
  {steady\_state\_hyper}
  {mV}
  {none}
  {V\\&T\\&stim end}
  {none}
  {Steady state voltage during hyperpolarization}
  {
  end\_index = minimal $i$ WITH t$_i \ge$ stim\_end \\
  mean = mean of V[$i$] WITH end\_index - 35 $\le i <$ end\_index - 5 \\
  APPEND mean TO steady\_state\_hyper
  }
  Find the voltage trace index at \myid{stim end}.
  Yield the average of \myid{V} between that index minus 35 and that index minus 5.
  
\end{efeature}

\begin{efeature}
  {amp drop first second}
  {LibV2}
  {amp\_drop\_first\_second}
  {double}
  {peak voltage}
  {none}
  {none}
  {Difference of the amplitude of the first and the second peak}
  {
  IF length of peak\_voltage < 2 THEN \+ \\
    FAIL "At least 2 spikes needed for the calculation of amp\_drop\_first\_second." \- \\
  ENDIF \\
  APPEND peak\_voltage[0] - peak\_voltage[1] TO amp\_drop\_first\_second
  }
  
\end{efeature}

\begin{efeature}
  {amp drop first last}
  {LibV2}
  {amp\_drop\_first\_last}
  {double}
  {peak voltage}
  {none}
  {none}
  {Difference of the amplitude of the first and the last peak}
  {
  IF length of peak\_voltage < 2 THEN \+ \\
    FAIL "At least 2 spikes needed for the calculation of amp\_drop\_first\_last." \- \\
  ENDIF \\
  APPEND peak\_voltage[0] - peak\_voltage[last] TO amp\_drop\_first\_last
  }
  
\end{efeature}

\begin{efeature}
  {amp drop second last}
  {LibV2}
  {amp\_drop\_second\_last}
  {double}
  {peak voltage}
  {none}
  {none}
  {Difference of the amplitude of the second and the last peak}
  {
  IF length of peak\_voltage < 3 THEN \+ \\
    FAIL "At least 3 spikes needed for the calculation of amp\_drop\_second\_last." \- \\
  ENDIF \\
  APPEND peak\_voltage[1] - peak\_voltage[last] TO amp\_drop\_second\_last
  }
  
\end{efeature}

\begin{efeature}
  {max amp difference}
  {LibV2}
  {max\_amp\_difference}
  {double}
  {peak voltage}
  {none}
  {none}
  {Maximum difference of the height of two subsequent peaks}
  {
  IF length of peak\_voltage < 2 THEN \+ \\
    FAIL "At least 2 spikes needed for the calculation of max\_amp\_difference." \- \\
  ENDIF \\
  APPEND $\max \Delta$peak\_voltage TO max\_amp\_difference
  }
  
\end{efeature}
\section{Regular features}

% efeature
% args:
%   feature name
%   namespace
%   identifier
%   type
%   required features
%   required trace data
%   required parameters
%   semantics
%   detailed description / specification

\begin{efeature}
  {back-propagating AP attenuation}
  {LibV2}
  {BPAPatt2}
  {none}
  {peak voltage;location\_soma\\&voltage base;location*}
  {V;location\_dend620}
  {none}
  {Ratio of relative heights of somatic peak and dendritic peak}
  {
  bpapatt = (peak\_voltage[0];location\_soma - voltage\_base;location\_soma) / ($\max_i$ V[$i$];location\_dend - voltage\_base;location\_dend) \\
  APPEND bpapatt TO BPAPatt2
  }
  Inject a short square pulse at soma that invokes exactly one action potential.
  The relative height of the action potential at soma is the difference of \myid{peak voltage;location\_soma} and \myid{voltage base;location\_soma}.
  The recording at the dendrite takes place at the thickest apical dendrite with a distance of 620 $\mu m$ from soma.
  The relative height is the difference of the maximum of \myid{V;location\_dend620} and \myid{voltage base;location\_dend620}.
  Yield the relative height at soma divided by the relative height at the dendritic location.
  
  \mybox{remarks:}{
  There exist a \emph{hoc} implementation under the same name, where instead of a ratio the actual relative height of the peak at the dendrite is returned.
  }
\end{efeature}

\begin{efeature}
  {back-propagating AP attenuation (2nd)}
  {LibV2}
  {BPAPatt3}
  {none}
  {peak voltage;location\_soma\\&voltage base;location*}
  {V;location\_dend800}
  {none}
  {Ratio of relative heights of somatic peak and dendritic peak}
  {
  bpapatt = (peak\_voltage[0];location\_soma - voltage\_base;location\_soma) / ($\max_i$ V[$i$];location\_dend - voltage\_base;location\_dend) \\
  APPEND bpapatt TO BPAPatt3
  }
  Inject a short square pulse at soma that invokes exactly one action potential.
  The relative height of the action potential at soma is the difference of \myid{peak voltage;location\_soma} and \myid{voltage base;location\_soma}.
  The recording at the dendrite takes place at the thickest apical dendrite with a distance of 800 $\mu m$ from soma.
  The relative height is the difference of the maximum of \myid{V;location\_dend800} and \myid{voltage base;location\_dend800}.
  Yield the relative height at soma divided by the relative height at the dendritic location.
  
\end{efeature}

\begin{efeature}
  {E2}
  {LibV2}
  {E2}
  {mV}
  {amp\_drop\_first\_second;APDrop*}
  {none}
  {none}
  {Difference of the amplitude of the first and the second peak}
  {
  APPEND mean of amp\_drop\_first\_second;APDrop* TO E2
  }
  Take the mean of \myid{peak voltage} over all repetitions of the stimulus protocol \myid{APDrop} for the first and the second peak.
  Yield the difference.
  
\end{efeature}

\begin{efeature}
  {E3}
  {LibV2}
  {E3}
  {mV}
  {amp\_drop\_first\_last;APDrop*}
  {none}
  {none}
  {Difference of the amplitude of the first and the last peak}
  {
  APPEND mean of amp\_drop\_first\_last;APDrop* TO E3
  }
  Take the mean of \myid{peak voltage} over all repetitions of the stimulus protocol \myid{APDrop} for the first and the last peak.
  Yield the difference.
  
\end{efeature}

\begin{efeature}
  {E4}
  {LibV4}
  {E4}
  {mV}
  {amp\_drop\_second\_last;APDrop*}
  {none}
  {none}
  {Difference of the amplitude of the second and the last peak}
  {
  APPEND mean of amp\_drop\_second\_last;APDrop* TO E4
  }
  Take the mean of \myid{peak voltage} over all repetitions of the stimulus protocol \myid{APDrop} for the second and the last peak.
  Yield the difference.

\end{efeature}

\begin{efeature}
  {E5}
  {LibV2}
  {E5}
  {mV}
  {max\_amp\_difference;APDrop*}
  {none}
  {none}
  {Maximum difference of the height of two subsequent peaks}
  {
  APPEND mean of max\_amp\_difference;APDrop* TO E5
  }
  Take the mean of \myid{peak voltage} over all repetitions of the stimulus protocol \myid{APDrop} for each peak.
  Yield the biggest difference between two peaks.
  
\end{efeature}

\begin{efeature}
  {E6 (AP amplitude)}
  {LibV2}
  {E6}
  {mV}
  {AP amplitude;APWaveForm*}
  {none}
  {none}
  {Relative height of the first action potential}
  {
  APPEND mean of AP\_amplitude;APWaveForm* at index 0 TO E6
  }
  Take the mean of the first \myid{AP amplitude} over all repetitions and iterations of the stimulus protocol \myid{APWaveForm}.
  
\end{efeature}

\begin{efeature}
  {E7 (AP duration)}
  {LibV2}
  {E7}
  {ms}
  {AP duration;APWaveForm*}
  {none}
  {none}
  {Duration of the first action potential}
  {
  APPEND mean of AP\_duration;APWaveForm* at index 0 TO E7
  }
  Take the mean of the first \myid{AP duration} over all repetitions and iterations of the stimulus protocol \myid{APWaveForm}.
  
\end{efeature}

\begin{efeature}
  {E8 (AP duration half width)}
  {LibV2}
  {E8}
  {ms}
  {AP duration half width;APWaveForm*}
  {none}
  {none}
  {FWHM of the first action potential}
  {
  APPEND mean of AP\_duration\_half\_width;APWaveForm* at index 0 TO E8
  }
  Take the mean of the first \myid{AP duration half width} over all repetitions and iterations of the stimulus protocol \myid{APWaveForm}.
  
\end{efeature}

\begin{efeature}
  {E9 (AP rise time)}
  {LibV2}
  {E9}
  {ms}
  {AP rise time;APWaveForm*}
  {none}
  {none}
  {Time from onset of the first action potential to the maximum}
  {
  APPEND mean of AP\_rise\_time;APWaveForm* at index 0 TO E9
  }
  Take the mean of the first \myid{AP rise time} over all repetitions and iterations of the stimulus protocol \myid{APWaveForm}.
  
\end{efeature}

\begin{efeature}
  {E10 (AP fall time)}
  {LibV2}
  {E10}
  {ms}
  {AP fall time;APWaveForm*}
  {none}
  {none}
  {Time from maximum of the first action potential to offset}
  {
  APPEND mean of AP\_fall\_time;APWaveForm* at index 0 TO E10
  }
  Take the mean of the first \myid{AP fall time} over all repetitions and iterations of the stimulus protocol \myid{APWaveForm}.
  
\end{efeature}

\begin{efeature}
  {E11 (AP rise rate)}
  {LibV2}
  {E11}
  {$\frac{\mathrm{V}}{\mathrm{s}}$}
  {AP rise rate;APWaveForm*}
  {none}
  {none}
  {Voltage change rate during the rising phase of the first action potential}
  {
  APPEND mean of AP\_rise\_rate;APWaveForm* at index 0 TO E11
  }
  Take the mean of the first \myid{AP rise rate} over all repetitions and iterations of the stimulus protocol \myid{APWaveForm}.
  
\end{efeature}

\begin{efeature}
  {E12 (AP fall rate)}
  {LibV2}
  {E12}
  {$\frac{\mathrm{V}}{\mathrm{s}}$}
  {AP fall rate}
  {none}
  {none}
  {Voltage change rate during the falling phase of the first action potential}
  {
  APPEND mean of AP\_fall\_rate;APWaveForm* at index 0 TO E12
  }
  Take the mean of the first \myid{AP fall rate} over all repetitions and iterations of the stimulus protocol \myid{APWaveForm}.
  
\end{efeature}

\begin{efeature}
  {E13 (fast AHP)}
  {LibV2}
  {E13}
  {mV}
  {fast AHP;APWaveForm*}
  {none}
  {none}
  {Voltage value of the onset of the first action potential relative to the subsequent AHP}
  {
  APPEND mean of fast\_AHP;APWaveForm* at index 0 TO E13
  }
  Take the mean of the first \myid{fast AHP} over all repetitions and iterations of the stimulus protocol \myid{APWaveForm}.
  
\end{efeature}

\begin{efeature}
  {E14 (AP amplitude)}
  {LibV2}
  {E14}
  {mV}
  {AP Amplitude;APWaveForm*}
  {none}
  {none}
  {Relative height of the second action potential}
  {
  APPEND mean of AP\_amplitude;APWaveForm* at index 1 TO E14
  }
  Take the mean of the second \myid{AP amplitude} over all repetitions and iterations of the stimulus protocol \myid{APWaveForm}.
  
\end{efeature}

\begin{efeature}
  {E15 (AP duration)}
  {LibV2}
  {E15}
  {ms}
  {AP duration;APWaveForm*}
  {none}
  {none}
  {Duration of the second action potential}
  {
  APPEND mean of AP\_duration;APWaveForm* at index 1 TO E15
  }
  Take the mean of the second \myid{AP duration} over all repetitions and iterations of the stimulus protocol \myid{APWaveForm}.
  
\end{efeature}

\begin{efeature}
  {E16 (AP duration half width)}
  {LibV2}
  {E16}
  {ms}
  {AP duration half width;APWaveForm*}
  {none}
  {none}
  {FWHM of the second action potential}
  {
  APPEND mean of AP\_duration\_half\_width;APWaveForm* at index 1 TO E16
  }
  Take the mean of the second \myid{AP duration half width} over all repetitions and iterations of the stimulus protocol \myid{APWaveForm}.
  
\end{efeature}

\begin{efeature}
  {E17 (AP rise time)}
  {LibV2}
  {E17}
  {ms}
  {AP rise time;APWaveForm*}
  {none}
  {none}
  {Time from onset of the second action potential to the maximum}
  {
  APPEND mean of AP\_rise\_time;APWaveForm* at index 1 TO E17
  }
  Take the mean of the second \myid{AP rise time} over all repetitions and iterations of the stimulus protocol \myid{APWaveForm}.
  
\end{efeature}

\begin{efeature}
  {E18 (AP fall time)}
  {LibV2}
  {E18}
  {ms}
  {AP fall time;APWaveForm*}
  {none}
  {none}
  {Time from maximum of the second action potential to offset}
  {
  APPEND mean of AP\_fall\_time;APWaveForm* at index 1 TO E18
  }
  Take the mean of the second \myid{AP fall time} over all repetitions and iterations of the stimulus protocol \myid{APWaveForm}.
  
\end{efeature}

\begin{efeature}
  {E19 (AP rise rate)}
  {LibV2}
  {E19}
  {$\frac{\mathrm{V}}{\mathrm{s}}$}
  {AP rise rate;APWaveForm*}
  {none}
  {none}
  {Voltage change rate during the rising phase of the second action potential}
  {
  APPEND mean of AP\_rise\_rate;APWaveForm* at index 1 TO E19
  }
  Take the mean of the second \myid{AP rise rate} over all repetitions and iterations of the stimulus protocol \myid{APWaveForm}.
  
\end{efeature}

\begin{efeature}
  {E20 (AP fall rate)}
  {LibV2}
  {E20}
  {$\frac{\mathrm{V}}{\mathrm{s}}$}
  {AP fall rate}
  {none}
  {none}
  {Voltage change rate during the falling phase of the second action potential}
  {
  APPEND mean of AP\_fall\_rate;APWaveForm* at index 1 TO E20
  }
  Take the mean of the second \myid{AP fall rate} over all repetitions and iterations of the stimulus protocol \myid{APWaveForm}.
  
\end{efeature}

\begin{efeature}
  {E21 (fast AHP)}
  {LibV2}
  {E21}
  {mV}
  {fast AHP;APWaveForm*}
  {none}
  {none}
  {Voltage value of the onset of the second action potential relative to the subsequent AHP}
  {
  APPEND mean of fast\_AHP;APWaveForm* at index 1 TO E21
  }
  Take the mean of the second \myid{fast AHP} over all repetitions and iterations of the stimulus protocol \myid{APWaveForm}.
  
\end{efeature}

\begin{efeature}
  {E22 (AP amplitude change)}
  {LibV2}
  {E22}
  {none}
  {AP amplitude change;APWaveForm*}
  {none}
  {none}
  {Difference of the amplitudes of the second and the first action potential divided by the amplitude of the first action potential}
  {
  APPEND mean of AP\_amplitude\_change;APWaveForm* at index 0 TO E22
  }
  Take the mean of \myid{AP amplitude change} over all repetitions and iterations of the stimulus protocol \myid{APWaveForm}.
  
\end{efeature}

\begin{efeature}
  {E23 (AP duration change)}
  {LibV2}
  {E23}
  {none}
  {AP duration change;APWaveForm*}
  {none}
  {none}
  {Difference of the durations of the second and the first action potential divided by the duration of the first action potential}
  {
  APPEND mean of AP\_duration\_change;APWaveForm* at index 0 TO E23
  }
  Take the mean of \myid{AP duration change} over all repetitions and iterations of the stimulus protocol \myid{APWaveForm}.
  
\end{efeature}

\begin{efeature}
  {E24 (AP duration half width change)}
  {LibV2}
  {E24}
  {none}
  {AP duration half width change;APWaveForm*}
  {none}
  {none}
  {Difference of the FWHM of the second and the first action potential divided by the FWHM of the first action potential}
  {
  APPEND mean of AP\_duration\_half\_width\_change;APWaveForm* at index 0 TO E24
  }
  Take the mean of \myid{AP duration half width change} over all repetitions and iterations of the stimulus protocol \myid{APWaveForm}.
  
\end{efeature}

\begin{efeature}
  {E25 (AP rise rate change)}
  {LibV2}
  {E25}
  {none}
  {AP rise rate change;APWaveForm*}
  {none}
  {none}
  {Difference of the rise rates of the second and the first action potential divided by the rise rate of the first action potential}
  {
  APPEND mean of AP\_rise\_rate\_change;APWaveForm* at index 0 TO E25
  }
  Take the mean of \myid{AP rise rate change} over all repetitions and iterations of the stimulus protocol \myid{APWaveForm}.
  
\end{efeature}

\begin{efeature}
  {E26 (AP fall rate change)}
  {LibV2}
  {E26}
  {none}
  {AP fall rate change;APWaveForm*}
  {none}
  {none}
  {Difference of the fall rates of the second and the first action potential divided by the fall rate of the first action potential}
  {
  APPEND mean of AP\_fall\_rate\_change;APWaveForm* at index 0 TO E26
  }
  Take the mean of \myid{AP fall rate change} over all repetitions and iterations of the stimulus protocol \myid{APWaveForm}.
  
\end{efeature}

\begin{efeature}
  {E27 (fast AHP change)}
  {LibV2}
  {E27}
  {none}
  {fast AHP change;APWaveForm*}
  {none}
  {none}
  {Difference of the \myid{fast AHP} of the second and the first action potential divided by the \myid{fast AHP} of the first action potential}
  {
  APPEND mean of fast\_AHP\_change;APWaveForm* at index 0 TO E27
  }
  Take the mean of \myid{fast AHP change} over all repetitions and iterations of the stimulus protocol \myid{APWaveForm}.
  
\end{efeature}

\begin{efeature}
  {E39}
  {LibV2}
  {E39}
  {$\frac{\mathrm{Hz}}{\mathrm{nA}}$}
  {mean frequency;IDthreshold*}
  {stimulus current;IDthreshold*}
  {none}
  {The slope of a linear fit of the curve \myid{mean frequency} vs. \myid{stimulus current}}
  {
  FOR suffix IN IDthreshold* DO \+ \\
    APPEND stimulus\_current;suffix TO x \\
    APPEND mean\_frequency;suffix TO y \- \\
  ENDFOR \\
  slope, residuals, $R^2$ = fit\_straight\_line(x, y) \\
  APPEND slope TO E39
  }
  Get the points (\myid{stimulus current}, \myid{mean frequency}) for all repetitions and iterations of the stimulus protocol \myid{IDthreshold} and fit a straight line.

  \mybox{remarks:}{
  A straight line generally is not an appropriate fit to the described curve.
  }
\end{efeature}

\begin{efeature}
  {E39: coefficient of determination}
  {LibV2}
  {E39\_cod}
  {none}
  {E39}
  {none}
  {none}
  {The coefficient of determination (often: R$^2$) of the straight line fit according to E39}
  {
  FOR suffix IN IDthreshold* DO \+ \\
    APPEND stimulus\_current;suffix TO x \\
    APPEND steady\_state\_hyper;suffix TO y \- \\
  ENDFOR \\
  slope, residuals, $R^2$ = fit\_straight\_line(x, y) \\
  APPEND $R^2$ TO E39\_cod
  }
  
\end{efeature}

\begin{efeature}
  {E40 (time to first spike)}
  {LibV2}
  {E40}
  {ms}
  {time to first spike;IDrest*}
  {none}
  {none}
  {Average time from the begin of the stimulus to the maximum of the first peak}
  {
  APPEND mean of time\_to\_first\_spike;IDrest* at index 0 TO E40
  }
  Take the mean of \myid{time to first spike} over all repetitions and iterations of the stimulus protocol \myid{IDrest}
  
\end{efeature}

\section{LibV5}

\begin{efeature}
  {AP begin voltage}
  {LibV5}
  {AP\_begin\_voltage}
  {mV}
  {AP\_begin\_indices}
  {V\\&T\\&stim start\\&stim end}
  {none}
  {Voltage values at the onset of each action potential}
  {
  begin$_0, \ldots, $begin$_{n-1} =$ AP\_begin\_indices \\
  FOR $i = 0, \dots, n - 1$ DO \+ \\
    APPEND V[begin$_{i}$] TO AP\_begin\_voltage \- \\
  ENDFOR
  }
  Return the voltage levels at \myid{AP\_begin\_indices}.
  
\end{efeature}

\begin{efeature}
  {AHP time from peak}
  {LibV5}
  {AHP\_time\_from\_peak}
  {ms}
  {min\_AHP\_indices\\&peak\_indices}
  {V\\&T\\&stim start\\&stim end}
  {none}
  {Time between AP peaks and AHP depths}
  {
  peak$_0, \ldots, $peak$_{m-1} =$ peak\_indices \\
  ahp$_0, \ldots, $ahp$_{n-1} =$ min\_AHP\_indices \\
  IF $m$ > $n$ THEN \+ \\
    FAIL \- \\
  ENDIF \\
  FOR $i = 0, \dots, m - 1$ DO \+ \\
    APPEND T[ahp$_i$] $-$ T[peak$_i$] TO AHP\_time\_from\_peak \- \\
  ENDFOR
  }
  Obtain the \myid{min\_AHP\_indices} and \myid{peak\_indices}, and calculate
  the time between these indices in the T array.
  
\end{efeature}

\begin{efeature}
  {AP amplitude from voltagebase}
  {LibV1}
  {AP\_amplitude\_from\_voltagebase}
  {mV}
  {voltage base\\&peak voltage}
  {V}
  {none}
  {The height of the action potential measured from voltage base}
  {
  pv$_0, \ldots, $pv$_{n-1} =$ peak\_voltage \\
  FOR $i = 0, \dots, n - 1$ DO \+ \\
    APPEND pv$_i$ - voltage\_base TO AP\_amplitude\_from\_voltagebase \- \\
  ENDFOR
  }
  Yield the difference of \myid{peak voltage} and \myid{voltage base} for each peak.
  
\end{efeature}
